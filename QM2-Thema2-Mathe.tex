% Options for packages loaded elsewhere
\PassOptionsToPackage{unicode}{hyperref}
\PassOptionsToPackage{hyphens}{url}
%
\documentclass[
  ngerman,
  ignorenonframetext,
]{beamer}
\usepackage{pgfpages}
\setbeamertemplate{caption}[numbered]
\setbeamertemplate{caption label separator}{: }
\setbeamercolor{caption name}{fg=normal text.fg}
\beamertemplatenavigationsymbolsempty
% Prevent slide breaks in the middle of a paragraph
\widowpenalties 1 10000
\raggedbottom
\setbeamertemplate{part page}{
  \centering
  \begin{beamercolorbox}[sep=16pt,center]{part title}
    \usebeamerfont{part title}\insertpart\par
  \end{beamercolorbox}
}
\setbeamertemplate{section page}{
  \centering
  \begin{beamercolorbox}[sep=12pt,center]{part title}
    \usebeamerfont{section title}\insertsection\par
  \end{beamercolorbox}
}
\setbeamertemplate{subsection page}{
  \centering
  \begin{beamercolorbox}[sep=8pt,center]{part title}
    \usebeamerfont{subsection title}\insertsubsection\par
  \end{beamercolorbox}
}
\AtBeginPart{
  \frame{\partpage}
}
\AtBeginSection{
  \ifbibliography
  \else
    \frame{\sectionpage}
  \fi
}
\AtBeginSubsection{
  \frame{\subsectionpage}
}
\usepackage{amsmath,amssymb}
\usepackage{lmodern}
\usepackage{iftex}
\ifPDFTeX
  \usepackage[T1]{fontenc}
  \usepackage[utf8]{inputenc}
  \usepackage{textcomp} % provide euro and other symbols
\else % if luatex or xetex
  \usepackage{unicode-math}
  \defaultfontfeatures{Scale=MatchLowercase}
  \defaultfontfeatures[\rmfamily]{Ligatures=TeX,Scale=1}
\fi
\usetheme[]{Berkeley}
% Use upquote if available, for straight quotes in verbatim environments
\IfFileExists{upquote.sty}{\usepackage{upquote}}{}
\IfFileExists{microtype.sty}{% use microtype if available
  \usepackage[]{microtype}
  \UseMicrotypeSet[protrusion]{basicmath} % disable protrusion for tt fonts
}{}
\makeatletter
\@ifundefined{KOMAClassName}{% if non-KOMA class
  \IfFileExists{parskip.sty}{%
    \usepackage{parskip}
  }{% else
    \setlength{\parindent}{0pt}
    \setlength{\parskip}{6pt plus 2pt minus 1pt}}
}{% if KOMA class
  \KOMAoptions{parskip=half}}
\makeatother
\usepackage{xcolor}
\IfFileExists{xurl.sty}{\usepackage{xurl}}{} % add URL line breaks if available
\IfFileExists{bookmark.sty}{\usepackage{bookmark}}{\usepackage{hyperref}}
\hypersetup{
  pdftitle={Thema 2: Mathematische Grundlagen},
  pdfauthor={Prof.~Sauer},
  pdflang={de-DE},
  hidelinks,
  pdfcreator={LaTeX via pandoc}}
\urlstyle{same} % disable monospaced font for URLs
\newif\ifbibliography
\setlength{\emergencystretch}{3em} % prevent overfull lines
\providecommand{\tightlist}{%
  \setlength{\itemsep}{0pt}\setlength{\parskip}{0pt}}
\setcounter{secnumdepth}{-\maxdimen} % remove section numbering
%\setbeamertemplate{page number in head/foot}[totalframenumber]
\setbeamertemplate{footline}[frame number]
\ifXeTeX
  % Load polyglossia as late as possible: uses bidi with RTL langages (e.g. Hebrew, Arabic)
  \usepackage{polyglossia}
  \setmainlanguage[]{german}
\else
  \usepackage[main=ngerman]{babel}
% get rid of language-specific shorthands (see #6817):
\let\LanguageShortHands\languageshorthands
\def\languageshorthands#1{}
\fi
\ifLuaTeX
  \usepackage{selnolig}  % disable illegal ligatures
\fi
\newlength{\cslhangindent}
\setlength{\cslhangindent}{1.5em}
\newlength{\csllabelwidth}
\setlength{\csllabelwidth}{3em}
\newenvironment{CSLReferences}[2] % #1 hanging-ident, #2 entry spacing
 {% don't indent paragraphs
  \setlength{\parindent}{0pt}
  % turn on hanging indent if param 1 is 1
  \ifodd #1 \everypar{\setlength{\hangindent}{\cslhangindent}}\ignorespaces\fi
  % set entry spacing
  \ifnum #2 > 0
  \setlength{\parskip}{#2\baselineskip}
  \fi
 }%
 {}
\usepackage{calc}
\newcommand{\CSLBlock}[1]{#1\hfill\break}
\newcommand{\CSLLeftMargin}[1]{\parbox[t]{\csllabelwidth}{#1}}
\newcommand{\CSLRightInline}[1]{\parbox[t]{\linewidth - \csllabelwidth}{#1}\break}
\newcommand{\CSLIndent}[1]{\hspace{\cslhangindent}#1}

\title{Thema 2: Mathematische Grundlagen}
\subtitle{QM2, ROS, Kap. 3}
\author{Prof.~Sauer}
\date{WiSe 21}
\institute{AWM, HS Ansbach}

\begin{document}
\frame{\titlepage}

\begin{frame}[allowframebreaks]
  \tableofcontents[hideallsubsections]
\end{frame}
\hypertarget{multiplikative-modelle}{%
\section{Multiplikative Modelle}\label{multiplikative-modelle}}

\begin{frame}{``Normale'' Regression: additiv, linear}
\protect\hypertarget{normale-regression-additiv-linear}{}
\[y = b0 + b_1x_1 + b_2 x_2 + \ldots + b_k x_k + \epsilon\]
\end{frame}

\begin{frame}{LogY-Regression: Exponenzielles Wachstum}
\protect\hypertarget{logy-regression-exponenzielles-wachstum}{}
\(log(y) = a + bx\)

\(y= Ae^{bx}\) mit \(A=e^a\)

\(e\) ist die Eulersche Zahl, \(2.71...\)
\end{frame}

\begin{frame}{Beispiele für exponentielle Zusammenhänge}
\protect\hypertarget{beispiele-fuxfcr-exponentielle-zusammenhuxe4nge}{}
\begin{itemize}
\tightlist
\item
  Eine Bakterienmenge verdoppelt sich jeden Tag
\item
  Pro Jahr erzielt eine Kapitalanlage 10\% Zinsen
\item
  Während einer bestimmten Periode verdoppelten sich die Coronafälle
  alle 10 Tage
\item
  Die Menge der Vitamine in einem Lebensmittel verringert sich pro
  Zeiteinheit um den Faktor \(k\)
\end{itemize}
\end{frame}

\begin{frame}{Exponentielles Wachstum wächst stark}
\protect\hypertarget{exponentielles-wachstum-wuxe4chst-stark}{}
Beim exponentiellen Wachstum wächst eine Größe pro Zeitabschnitt immer
um denselben Faktor.

Die Änderung einer Größe \(A\) pro Zeitabschnitt \(t\) ist
proportional\footnote<.->{Proportional bedeutet, Verdopplung
  (Verdreifachung, Vervierfachung\ldots) einer Größe ist stets mit der
  Verdopplung (Verdreifachung, Vervierfachung, \ldots) der anderen Größe
  verbunden. So ist der Kreisumfang proportional zum Kreisdurchmesser
  mit dem Proportionalitätsfaktor 3.14.} zum Bestand von \(A\).

Exponentielles Wachstum wächst (ab einem bestimmten Zeitpunkt) sehr
stark und wird daher leicht unterschätzt.
\end{frame}

\begin{frame}{Aber was ist der Logarithmus?}
\protect\hypertarget{aber-was-ist-der-logarithmus}{}
\begin{alertblock}{Definition}
Seien $a,b > 0$ mit $b \ne 1$.

Die eindeutig bestimmte Zahl $x \in \mathbb{R}$ 
 mit $b^x=a$ heißt Logarithmus von $a$ zur Basis $b$. Sie wird mit $x = log_b(a)$ bezeichnet.

\end{alertblock}

(Cramer und Nešlehová 2015.)
\end{frame}

\begin{frame}{Beispiele zum Logarithmus}
\protect\hypertarget{beispiele-zum-logarithmus}{}
\begin{columns}[T]
\begin{column}{0.48\textwidth}
\begin{block}{Basis 10}
\protect\hypertarget{basis-10}{}
\begin{itemize}
\tightlist
\item
  \(\log_{10}(10) = 1\)
\item
  \(\log_{10}(100) = 2\)
\item
  \(\log_{10}(1000) = 3\)
\end{itemize}
\end{block}
\end{column}

\begin{column}{0.48\textwidth}
\begin{block}{Basis 2}
\protect\hypertarget{basis-2}{}
\begin{itemize}
\tightlist
\item
  \(\log_{2}(2) = 1\)
\item
  \(\log_{2}(4) = 2\)
\item
  \(\log_{2}(8) = 3\)
\end{itemize}
\end{block}
\end{column}
\end{columns}
\end{frame}

\hypertarget{hinweise}{%
\section{Hinweise}\label{hinweise}}

\begin{frame}{Lehrbuch und Homepage des Lehrbuchs}
\protect\hypertarget{lehrbuch-und-homepage-des-lehrbuchs}{}
Dieses Skript bezieht sich auf ausgewählte Kapitel dieses Lehrbuchs:
Gelman, Hill, und Vehtari (2021)

Das Buch ist im Skript mit ``ROS'' abgekürzt.

Weitere Literaturhinweise sind am Ende der jeweiligen Kapitel des
Lehrbuchs angegeben.

R-Code zum Buch findet sich auf der
\href{https://avehtari.github.io/ROS-Examples/examples.html}{Homepage}
des Buchs.
\end{frame}

\begin{frame}{Literatur}
\protect\hypertarget{literatur}{}
\hypertarget{refs}{}
\begin{CSLReferences}{1}{0}
\leavevmode\vadjust pre{\hypertarget{ref-cramer_vorkurs_2015}{}}%
Cramer, Erhard, und Johanna Nešlehová. 2015. \emph{Vorkurs Mathematik}.
{Berlin, Heidelberg}: {Springer Berlin Heidelberg}.
\url{https://doi.org/10.1007/978-3-662-46400-7}.

\leavevmode\vadjust pre{\hypertarget{ref-gelman_regression_2021}{}}%
Gelman, Andrew, Jennifer Hill, und Aki Vehtari. 2021. \emph{Regression
and Other Stories}. Analytical Methods for Social Research. {Cambridge}:
{Cambridge University Press}.

\end{CSLReferences}
\end{frame}

\end{document}
